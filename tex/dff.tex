\documentclass[addpoints]{exam}
\pagestyle{headandfoot}

\usepackage[T1,T2A]{fontenc}
\usepackage[utf8]{inputenc}
\usepackage[russian]{babel}

\usepackage{amsmath, amsfonts}
\usepackage{verbatim}
\usepackage{graphicx}
\usepackage[usenames,dvipsnames]{color}
\usepackage{parskip}
\usepackage{tikz} % drawing schemes
\newcommand{\semester}{WS 2025/2026}
\runningheader{\students}{}{\semester}
\runningfooter{}{\thepage}{}
\headrule
\footrule

% ---------- Modify team name, students (matriculation number), exercise number here ----------
\newcommand{\teamname}{dl2023-team1}
\newcommand{\students}{Великанов Максим}
\newcommand{\assignmentnumber}{1}

\usepackage{tikz} % для создания иллюстраций
\newcommand{\framed}[1]{\tikz[baseline=(char.base)]{\node[shape=rectangle,draw,inner sep=4pt] (char) {#1};}}
% ---------- End Modify ----------

\title{Дифференциальное уравнение}
\author{Великанов Максим}
\date{\today}

\begin{document}
    \maketitle

    % ---------- Add Solution below here ----------
	
	Решаем дифференциальное уравнение \[ y' = y \Rightarrow \cfrac{dy}{dx} = y \]
	
	Слева - обычное деление приращения по y (так называемого дифференциала) на приращение по x.
	
	\textbf{Уравнение касательной} тоже имеет дело с малыми приращениями (удобно нарисовать): \[ \cfrac{y - y_0}{x - x_0} = f'(x_0) = \tg \alpha \]
	
	\textbf{1. Точное решение}
	
	Метод разделяемых переменных (y в одну сторону, x в другую):
	
	\[ \cfrac{dy}{dx} = y \Rightarrow \cfrac{dy}{y} = dx \]
	
	(не забыть y=0). Вспомним производную \( (\ln x)' = \cfrac{1}{x} \)
	
	\[ \int \cfrac{dy}{y} = \int dx \Rightarrow \ln y = x + c \Rightarrow \framed{ $y = e^{x + c} = e^x e^c$ } \]
	
	\textbf{2. Аппроксимация}
	
	По определению производной, и применяя исходное уравнение
	
	\[ \cfrac{y_1 - y_0}{x_1 - x_0} = y'(x_0) = y_0 \Rightarrow y_1 = \Delta x y_0 + y_0 \]
	
	Этот подход также называется дискретизацией дифференциала. Для вычисления решения нужно задать начальное условие (например \( x=0, y=1 \)).
	
	\begin{figure*}[h!]
		\centering
		\includegraphics[width=.7\linewidth]{img/solution.png}
		\caption{Решение \( y' = y \) при разных начальных условиях.}
		\label{fig:domain-communication}
	\end{figure*}

    % ---------- End of Document ----------
    

\end{document}
